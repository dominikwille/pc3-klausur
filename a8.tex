\section*{Aufgabe 8: Spektroelektrochemie (10)}
Kohlenstoffträgermaterialien sind korrosionsempfindlich und müssen zukünftig durch
stabilere materialien z.B. Oxide ersetzt werden. Für ihe Bachelorarbeit sollen sie
herausfinden wie \(CO\) an einem selbsthergestellten \(Pt/SnO_2\)-Katalysator adsorbiert.

\begin{enumerate}

\item Begründen sie einen Vorschlag für ein geeignetes spektroelektrochemoisches Experiment.
(3)

\item Was müssen sie beim Zelldesign beachten? (4)

\item Wofür steht die Abkürzung SECM? Erläutren sie wie Signale beim SECM zusatnde
kommen. (3)

\end{enumerate}