\section*{Aufgabe 2: Ionenleitung (13)}
\begin{enumerate}

\item Schreiben sie die Gleichung für ein Kräftegleichgewicht das auf ein Ion in Lösung
wirkt und benennen sie die vorkommenden Größen. (3)

\item Berechnen sie den Ionenradius für solvatisierte \(Cu^{2+}\)-Ionen in \(0,1\, M\)
\(HClO_4\), die eine Wanderungsgeschwindigkeit von \(1,04 \, cm/h\) bei einer Spannung
von \(1\, V\) und einem Elektronenabstand von \(1\, cm \) haben. Für die Viskosität
können sie näherungsweise die Viskosität von \(H_2O\) bei \(298\, K\) von \(0,891\, mPa\ s\) 
verwenden. (\(1\,mPa = 10^{-7}\, N/cm^2\)) \textbf{Einheitenrechnung!} (4)

\item Anstelle der Ionengeschwindigkeit verwendet man häufig die normierte größe 
,,Ionenbeweglichkeit``. beschreiben sie diese Größe und begründen sie weshalb diese
normiert wird. Bennennen sie ggf. vorkommende Größen in einer Gleichung. (4)

\item Begründen sie wie sich folgende Größen auf \(v_{max}\) von solvatisierten Ionen
auswirken. (2)
\begin{itemize}
\item Ein kleinerer Ionenradius des unsolvatisierten Ions.
\item Eine höhere Ionenladung.
\end{itemize}
\end{enumerate}