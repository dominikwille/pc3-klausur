\section*{Aufgabe 6: Elektrochemie im Gleichgewicht (9)}
\begin{enumerate}

\item Geben sie Daniell-Element die Nernstgleichung für die einzelnen Halbzellen sowie
für das gesamte Daniell-Element an. Bestimmen sie hiermit die elektromotorische Kraft. Wann ist die Nernstgleichung
gültig? Fließt dabei Strom? (4)

\item Skizzieren sie eine dafür geeignete elektrochemische Zelle und beschriften sie
diese vollständig. (2)

\item Welches System können sie für die zweite Halbzelle in einer galvanischen Zelle
wählen damit Silber abgeschieden wird? Begründen sie dies allgemein und geben sie
ein spezielles Beispiel an. Berechnen sie die elektrochemischen Gleichgewichtspotentiale
der folgenden Metallionelektroden unter Standardbedingungen. (5)
\begin{itemize}
\item[1.] \(Ag/Ag^+\) (\(c=0,001\, mol/l\))
\item[2.] \(Ag/Ag^+\) (\(c=0,01\, mol/l\))
\end{itemize}

\end{enumerate}